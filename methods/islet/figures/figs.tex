% hy figs-methods.hy basis-schematic 6
\begin{figure}[tbh]
  \centering
  \includegraphics[width=0.75\linewidth]{basis-schematic-np6-annotated}
  \caption{
    Basis functions for the Islet $\np=6$ GLL nodal subset basis listed in Table \ref{tbl:gll}.
    Each curve's color corresponds to a basis function.
    Each line pattern corresponds to a basis type, as listed in the legend.
    The green span shows region 1.
    The red arrows point to the nodes in the support of region 1;
    the red $\times$ is beneath the one node not in region 1's support.
  }
  \label{fig:np6-basis}
\end{figure}

\begin{table}[tbh]
  \input{figs/methods-table-gll.tex}
  \caption{
    Islet GLL nodal subset bases.
    Each row provides a formula for the row's $\np$ value.
    Columns are $\np$, order of accuracy (OOA),
    the support sizes $\npsub$ for each region ordered left to middle,
    and the supports.
    For offset nodal subset bases, supports are given by offsets.
    For general nodal subset bases, supports are given by nodal subsets, again ordered from left region to middle.
    The case $\np=4$ is described in Sect.~\ref{sec:np4}.
    In all cases, the support points are GLL points.
  }
  \label{tbl:gll}
\end{table}

% ./pum_sweep 8 512 4 0 > pum_sweep-np8-gll_natural.txt
% ./pum_sweep 8 512 4 1 > pum_sweep-np8-gll_best.txt
% ./pum_sweep 8 512 4 2 > pum_sweep-np8-uni.txt
% ./run_meam1_sweep 8 > run_meam1_sweep-np8.txt
% hy figs-methods.hy meam1-and-pum-vs-dx
\begin{figure}[tbh]
  \centering
  \includegraphics[width=0.5\linewidth]{meam1-and-pum-vs-dx}
  \caption{
    $\lambdamax(\Delta x)-1$ (solid lines) and $\lambdamaxpum(\Delta x)-1$ (markers) for
    the natural GLL (red, small circles), uniform-points offset nodal subset (green, $\times$), and
    Islet GLL nodal subset (black, large circle) $\np=8$ bases.
    Green dotted vertical lines mark multiples of $1/(\np-1)=1/7$.
  }
  \label{fig:meam1-and-pum-vs-dx}
\end{figure}

% ./pum_perturb_plot > pum_perturb_plot-041021.txt
% hy figs-methods.hy pum-vs-perturb
\begin{figure}[tbh]
  \centering
  \includegraphics[width=0.5\linewidth]{pum-vs-perturb}
  % include only gll_best b/c uniform_offset_nodal_subset was already shown to
  % be bad at i*1/(np-1) > 0.5, integer i.
  \caption{
    $\lambdamaxpum(\delta)-1$ for the bases in Table \ref{tbl:gll} with $\np \ge 6$.
    The triangle provides a $\delta^4$ reference slope.
  }
  \label{fig:pum-vs-perturb}
\end{figure}

% hy figs-methods.hy np4-schematic
\begin{figure}[tbh]
  \centering
  \includegraphics[width=0.5\linewidth]{np4-schematic}
  \caption{
  Illustration of the optimized Islet GLL $\np=4$ basis (solid line) compared with
  the natural (dotted) and the best nodal subset (dashed) $\np=4$ bases.
  Each basis function in a basis has its own color.
  The top panel shows the convex combination parameter value as a function of reference coordinate
  that is used to combine the natural and best nodal subset bases
  to form the optimized basis.
  }
  \label{fig:np4-schematic}
\end{figure}

% hy figs-methods.hy illustrations
\begin{figure}[tbh]
  \centering
  \includegraphics[width=0.5\linewidth]{illustrate-grids}
  \caption{
    One spectral element (blue solid line outlining the full square) with
    dynamics (black large circles), tracer (small red circles), and physics (green dashed lines) subelement grids.
  }
  \label{fig:illustrate-grids}
\end{figure}

% bash run-stability-cmp.sh > stability-cmp-0.txt
% hy figs-adv-diag.hy fig-stab-cmp stability-cmp-0.txt
\begin{figure}[tbh]
  \centering
  \includegraphics[width=0.5\linewidth]{stab-cmp-l2}
  \caption{
    Stability of the Islet method with the Islet GLL bases,
    compared with the instability of the method with the natural GLL bases.
    The $x$-axis is average dynamics grid point spacing at the equator in degrees for the quasiuniform cubed-sphere grid.
    The $y$-axis is $\log_{10} l_2$ relative error.
    A curve's line pattern corresponds to basis type and number of cycles, as listed in the top legend.
    A curve's marker corresponds to $\npt$, as listed in the bottom legend.
    The case is divergent flow, Gaussian hills ICs, property preservation, $p$-refinement, and long time steps.
  }
  \label{fig:islet-vs-gll}
\end{figure}

% bash run-accuracy.sh > acc-0.txt
% hy figs-adv-diag.hy fig-midpoint acc-0.txt
\begin{figure}[tbh]
  \centering
  \includegraphics[width=0.5\linewidth]{midpoint-check}
  \caption{
    Comparison of relative errors calculated at the test simulation's midpoint time of 6 days (1/2 cycle, dashed lines)
    and endpoint time of 12 days (1 cycle, solid lines).
    Each number at the right side of the plot is the empirical OOA computed using the final two points of the 1-cycle result.
  }
  \label{fig:traj-interp}
\end{figure}

% hy figs-adv-diag.hy figs-acc acc-0.txt
\begin{figure}[tbh]
  \centering
  \includegraphics[width=0.5\linewidth]{acc-nondivergent-gau-exact-nopp-fac5}
  \caption{
    Empirical verification of the order of accuracy of the Islet GLL bases.
    Each number at the right side of the plot is empirical OOA computed using the final two points of the $l_\infty$ curve.
  }
  \label{fig:islet-empirical-ooa}
\end{figure}

% hy figs-methods.hy write-slmmir-script
% bash run-slmmir-on-basis-lines.sh > slmmir-on-basis-lines-2.txt
% hy figs-methods.hy plot-slmmir-vs-heuristic # uses slmmir-on-basis-lines-2.txt
\begin{figure}[tbh]
  \centering
  \includegraphics[width=0.48\linewidth]{slmmir-vs-heuristic-gau-nopp-l2}
  \caption{$l_2$ norm on the nondivergent flow problem
    using basis $\basisns_{\np}$ vs.~$a_2(\basisns_{\np})$,
    for a large number of \abtps~bases and $\np=6$ to $10$.
    The legend lists the marker type for each $\np$.
    Large red circles outline the bases in Table \ref{tbl:gll}.
    The configuration uses the Gaussian hills IC and no property preservation.}
  \label{fig:slmmir-vs-heuristic-a}
\end{figure}
\begin{figure}[tbh]
  \centering
  \includegraphics[width=0.48\linewidth]{slmmir-vs-heuristic-cos-pp-l2}
  \caption{Same as Fig.~\ref{fig:slmmir-vs-heuristic-a} except that the configuration
    uses the cosine bells IC with property preservation.}
  \label{fig:slmmir-vs-heuristic-b}
\end{figure}

% hy figs-adv-diag.hy figs-acc acc-0.txt
\begin{figure}[tbh]
  \centering
  \includegraphics[width=0.48\linewidth]{acc-nondivergent-gau-interp-pp-fac1}
  \caption{
    Accuracy diagnostic.
    Compare with Figs.~1, 2 in TR14.
  }
  \label{fig:islet-acc-nondiv-gau-a}
\end{figure}
\begin{figure}[tbh]
  \centering
  \includegraphics[width=0.48\linewidth]{acc-nondivergent-gau-interp-pp-fac5}
  \caption{
    Accuracy diagnostic.
    Compare with Figs.~1, 2 in TR14.
  }
  \label{fig:islet-acc-nondiv-gau-b}
\end{figure}
\begin{figure}[tbh]
  \centering
  \includegraphics[width=0.48\linewidth]{acc-nondivergent-cos-interp-pp-fac1}
  \caption{
    Accuracy diagnostic.
    Compare with Fig.~3 in TR14.
  }
  \label{fig:islet-acc-nondiv-cos-a}
\end{figure}
\begin{figure}[tbh]
  \centering
  \includegraphics[width=0.48\linewidth]{acc-nondivergent-cos-interp-pp-fac5}
  \caption{
    Accuracy diagnostic.
    Compare with Fig.~3 in TR14.
  }
  \label{fig:islet-acc-nondiv-cos-b}
\end{figure}
\begin{figure}[tbh]
  \centering
  \includegraphics[width=0.48\linewidth]{acc-divergent-cos-interp-pp-fac1}
  \caption{
    Accuracy diagnostic.
    Compare with Fig.~16 in TR14.
  }
  \label{fig:islet-acc-div-cos-a}
\end{figure}
\begin{figure}[tbh]
  \centering
  \includegraphics[width=0.48\linewidth]{acc-divergent-cos-interp-pp-fac5}
  \caption{
    Accuracy diagnostic.
    Compare with Fig.~16 in TR14.
  }
  \label{fig:islet-acc-div-cos-b}
\end{figure}

% hy figs-adv-diag.hy fig-filament acc-0.txt
\begin{figure}[tbh]
  \centering
  \includegraphics[width=1\linewidth]{filament}
  \caption{
    Filament diagnostic, following Sect.~3.3 of TS12.
    Compare with Fig.~5 in TR14.
    The top row shows the diagnostic measured on the $\npv=4$ dynamics grid;
    the bottom row, on the tracer grid.
    The legend describes the dynamics-grid resolution and the time step length.
    The prescribed validation problem is the nondivergent flow with cosine bells IC.
    Property preservation is on.
    The $x$-axis is $\tau$, the mixing ratio threshold.
    The $y$-axis is the percent area having mixing ratio at least $\tau$ relative to that at the initial time.
  }
  \label{fig:filament}
\end{figure}

% bash run-mixing.sh > mixing-0.txt
% hy figs-adv-diag.hy figs-mixing mixing-0.txt
\begin{figure}[tbh]
  \centering
  \includegraphics[width=1\linewidth]{mixing-ne20.png}
  \caption{
    Mixing diagnostic, following Sect.~3.5 of TS12.
    Compare with Figs.~11--14 in TR14.
    This figure shows results for dynamics-grid resolution of 1.5$^\circ$.
    $l_o$ is exactly 0 in all cases because shape preservation is on, and so is not shown.
    See the text for further details.}
  \label{fig:mixing-ne20}
\end{figure}
\begin{figure}[tbh]
  \centering
  \includegraphics[width=1\linewidth]{mixing-ne40.png}
  \caption{Same as Fig.~\ref{fig:mixing-ne20} but with dynamics-grid resolution $0.75^\circ$.}
  \label{fig:mixing-ne40}
\end{figure}

% bash run-img-filament.sh > filament-imgs-0.txt
% hy figs-adv-diag.hy img-filament filament-imgs-0.txt filament-imgs
\begin{figure}[tbh]
  \centering
  \includegraphics[width=1\linewidth]{slo-midpoint}
  \caption{
    Images of the slotted cylinders IC advected by the nondivergent flow at the simulation's midpoint.
    Each column corresponds to a spatial resolution and time step length configuration,
    as stated at the top of each column.
    Each row corresponds to a particular value of $\npt$, as stated in the text at the top-right of each image.
    We omit $\npt=12$ results for the $0.75^\circ$ resolution because they are essentially identical at the resolution of the figure to the $\npt=8$ images.
  }
  \label{fig:slocyl-midpoint}
\end{figure}
\begin{figure}[tbh]
  \centering
  \includegraphics[width=1\linewidth]{slo-finpoint}
  \caption{
    Same as Fig.~\ref{fig:slocyl-midpoint} but for the simulation final point.
    Error measures are printed at the bottom-left of each image; see text for details.
  }
  \label{fig:slocyl-finpoint}
\end{figure}

% bash run-pg-srcterm-midpoint-test.sh > pg-srcterm-midpoint-test-nbdy3-1.txt
% hy figs-adv-diag.hy fig-pg-mimic-src-term pg-srcterm-midpoint-test-nbdy3-1.txt
% nbdy3 => edge_np = interior_np = 3
\begin{figure}[tbh]
  \centering
  % pg = np and pg = 2
  \includegraphics[width=0.5\linewidth]{acc-pg-mimic-src-term-midpoint-nondivergent-gau-interp-pp-fac5-l2}
  \caption{
    Validation of the remap of tendencies from physics to tracer grids and state from tracer to dynamics grids.
    See Sect.~\ref{sec:results:sources} for a description of the problem.
  }
  \label{fig:pg-mimic-src-term}
\end{figure}

% bash run-toychem-diagnostic.sh > toychem-diagnostic-nbdy3-0.txt
% hy figs-adv-diag.hy fig-toychem-diagnostic toychem-diagnostic-nbdy3-0.txt
\begin{figure}[tbh]
  \centering
  % pg = np-2
  \includegraphics[width=0.5\linewidth]{toychem-diagnostic}
  \caption{
    Toy chemistry diagnostic values as a function of time for ten cycles of the nondivergent flow.
    Time is on the $x$-axis and measured in cycles.
    Diagnostic values for the $l_2$-norm (solid lines) and $l_\infty$-norm (dashed lines) are on the $y$-axis.
    Markers as listed in the bottom legend are placed at the start of each cycle to differentiate the curves.
  }
  \label{fig:toychem-diagnostic}
\end{figure}

% bash run-toychem-imgs.sh
% hy figs-adv-diag.hy fig-toychem-finpoint toychem-imgs-nbdy3
\begin{figure}[tbh]
  \centering
  \includegraphics[width=1\linewidth]{toychem-finpoint}
  \caption{
    Images of the monatomic tracer at the end of the first cycle.
    Text at the lower left of each image states the configuration.
    Text at the upper right reports global extremal values.
  }
  \label{fig:toychem-finpoint}
\end{figure}
\begin{figure}[tbh]
  \centering
  \includegraphics[width=1\linewidth]{toychem-finpoint-diagnostic}
  \caption{
    Same as Fig.~\ref{fig:toychem-finpoint}, but now the images are of $(X_T - \bar{X}_T)/\bar{X}_T$.
  }
  \label{fig:toychem-finpoint-diagnostic}
\end{figure}

% bash run-isl-footprint.sh > isl-footprint-1.txt
% hy figs-adv-diag.hy fig-comm-footprint isl-footprint-1.txt
\begin{figure}[tbh]
  \centering
  \includegraphics[width=0.5\linewidth]{isl-footprint}
  \caption{
    Communication volume, in number of real scalars transmitted in $q$-messages
    per tracer per element per time step ($y$-axis)
    vs.~time in days of the simulation ($x$-axis),
    in the case of one element per process,
    for the nondivergent flow,
    with long (left) and short (right) time steps.
    Statistic and $\npt$ line patterns are stated in the legends.
  }
  \label{fig:footprint}
\end{figure}

% code branch: https://github.com/ambrad/E3SM/releases/tag/islet-2d-paper-summit-sl-gpu-timings
% data: https://github.com/E3SM-Project/perf-data/tree/main/nhxx-sl-summit-mar2021
% generate a table of data:
%   hy sl-gpu-perf.hy table "perf-data/nhxx-sl-summit-mar2021/data/qsize10/*"
%   hy sl-gpu-perf.hy table "perf-data/nhxx-sl-summit-mar2021/data/qsize40/*"
% we use these table entries to make the figure in addition to the SC20 paper's
% data:
% >>> ne 1024 qsize 10 nmax  4096 alg Eul    main_loop
% 1024  383.49   0.29
% 2048  225.43   0.50
% 4096  132.30   0.85
% 4600  120.84   0.93
% >>> ne 1024 qsize 10 nmax  4096 alg SL     main_loop
% 1024  253.64   0.44  1.51
% 2048  146.66   0.77  1.54
% 4096   89.18   1.26  1.48
% 4600   81.39   1.38  1.48
% >>> ne 1024 qsize 40 nmax  4096 alg Eul    main_loop
% 2048  461.20   0.24
% 4096  274.52   0.41
% 4600  257.60   0.44
% >>> ne 1024 qsize 40 nmax  4096 alg SL     main_loop
% 2048  167.22   0.67  2.76
% 4096   99.70   1.13  2.75
% 4600   90.23   1.24  2.85
% hy sl-gpu-perf.hy fig
\begin{figure}[tbh]
  \centering
  \includegraphics[width=0.5\linewidth]{sl-gpu-perf-032521-islet}
  \caption{
    Performance comparison of SL transport with $\npv=\npt=4$ vs.~Eulerian transport
    in the E3SM Atmosphere Model's dynamical core on the Summit supercomputer.
    The $x$-axis is number of NVIDIA V100 GPUs on Summit used in a run;
    the $y$-axis is dycore throughput reported in simulated years per wallclock day (SYPD).
    The black curves are for Eulerian transport; the red, for SL.
    Dashed lines are for 40 tracers; solid and the dotted black line, for 10.
    A number above a data point reports the $y$-value of that point.
  }
  \label{fig:summit-perf}
\end{figure}
